%!TEX root = project.tex

\chapter*{Abstract}
\paragraph{Abstract}
For my final year project, I was looking to create a project that would offer the chance for users to bet and lay on sporting events. A betting exchange is something I have become very interested in recent times and thought that this would be a great opportunity to research the subject and develop my own solution. I wanted to develop this as a web application where a user can create an account, top-up their account, place bets and also lay bets. The bets are populated and set via pulling data from other popular betting companies to get the best price possible. I created this application with a client and server pulling data from the data from the databases. It is a three-tier architecture application using a MySQL data base in the data tier, PHP on the logic tier and HTML and BootStrap for the Presentation tier.

\paragraph{Authors}
Colm Woodlock



\chapter{Introduction}
When choosing my project, I wanted to pick something that was relevant to what I was interested in and something that I might find useful. This project would also have to highlight my existing skills but also offered some space to learn new skills and research a topic I may not be fully familiar with. With this criteria in mind I started to brain storm ideas on some of the things that I might like to do. 

In the end I settled on a web application called BetEx. BetEx would be a betting exchange. My reason for choosing this is as a few years ago like many other college student was looking for some way to make some extra money on the side with as little effort as possible. That is when I stumbled across a site called OddsMonkey~\cite{oddsMonkey}, on this site they claimed if you followed their guides and tips a user could claim free money that bookmakers were simply giving away. Having seen the offers for "Bet €20 get €20" all over television advertisements, billboards, newspapers and websites, I always thought to myself there must be a way of taking advantage of this but never gave it much more thought. Upon reading their guide I gave it a go and low and behold I had turned my free €20 bet into €15 of take home money with minimal effort. From there I did more research discovering new methods, all the while reaping the benefits of them. This has funded both my education and various holidays. A key component to matched betting is the betting exchange and not many exist on a large scale BetFair, Smarkets and Matchbook being the largest of them~\cite{companies}. This lead me to the idea to develop my own platform to try and understand the workings behind the scenes of a betting exchange. I felt like there was an opportunity to make an exchange that had a much more user friendly interface as I found working with some of the established companies websites to be very confusing at the start.
Once the project idea was decided on I then had to think about what technologies I wanted to use. I wanted to use many different types of technologies in order to demonstrate my ability but also some that I would have to learn to show my capability to learn new things and incorporate them into an application. In the end I settled on a three tier structure containing a HTML and Boostrap presentation tier, a PHP logic tier which would offer new learning experiences and also for the data tier I would be using a MySQL database.
From all of this I hoped that I would achieve a product in the end that would meet my expectation of both myself and the expectations worthy of a final year project in a Software Development course.

\chapter{Context}
The context of my application is to offer a service to users who are new to laying bets. On the site a user can create an account, top-up their account and place a bet/lay from a list of real and regularly updated events.
\begin{itemize}
\item Log-in.
\item Top-up.
\item Place bet/lay.
\end{itemize}

\section{Objectives}
\subsection{Home Page}
The objective of the home page is to offer a landing platform for the user. Here they will be presented with the navigation for the website such as access to log in and register pages. A user who is not logged in may not have access to some other features presented on the home page.

\subsection{Register Page}
The register page offers a form to the user where they can securely create an account for the web application. Once the form is filled out correctly and successfully the user automatically sent out an email for confirmation.

\subsection{Log in Page}
The log in page allows users who have already registered for the site to access their account. Here they enter their email and their password which is stored as a hashed value. The logic behind the page then compares the password entered to the hashed value that is stored in the database. If the passwords match the user is logged in and they now have access to more features.

\subsection{Activation Page}
The activation page is used for when the user registering. A code will be automatically generated and emailed to the user to their provided email account and once entered the user will then be fully registered.

\subsection{Forgot Password Page}
Similar to the Activation page the user will be sent a code that is automatically generated and emailed to their provided email account. Once the user enters this code and it is correct they will then have the opportunity to change their password.

\subsection{Top up Page}
Provided the user is logged in they will have access to the top-up page. Here the user will be able to enter their details and the top-up amount will be added to their account to be used when betting/ laying on events.

\subsection{User Page}
The user page will be available once the user is logged in, here it will display information that is personal to that user such as their active bets or their account credits.

\subsection{Available Bets Page}
On the available bets page the user will be able to see the events that are currently available to bet/lay on. 

\subsection{Bet/Lay Page}
On this page, once the user has selected an event to bet on they will be given the opportunity to select an amount they wish to place a bet/lay on the event. Once it is placed it will be assigned to their account which they will be able to see on their user page.

\section{Project Links}
The full project is available here on GitHub:
https://github.com/cwoodlock/4thYearSoftwareProject

\section{Chapter Review}
This paper is broken down into various chapters ranging from the planning of the project to the design and implementation of a solution. The following sub-sections will give you a brief overview of each chapter in the paper.

\subsection{Methodology}
In this chapter I will discuss some of the methodologies I used during the various stages of creating the betting exchange web application. In this section I discuss Agile, version control, testing and sprints.

\subsection{Technology Review}
In the technology review chapter I review the various technologies that I used in the development of my project. I review everything from the back and front end technologies to development tools and version control.

\subsection{System Design}
In this chapter I give an explanation of the overall architecture of the betting exchange. I explain why I decided to do certain things and how I implemented those decisions into the project.
\subsection{System Evaluation}
In this chapter I evaluate the system in regards to robustness, testing, test results and limitations. 

\subsection{Conclusion}
Finally in the conclusion I summaries the project against what I set out to do in the goals and objectives. I review the application from the various stages of development and I talk about the possible future developments with the project.

\chapter{Methodology}
In this chapter i will discuss the methodologies used in this project. A methodology is just a way to plan and control the development process of a piece of software effectively. There are many different types of methodologies such as extreme programming or waterfall but for this project I tried to used Agile as my main methodology.

\section{Agile Development}
During the life cycle of this project I attempted to use agile as my main methodology whenever I could. I used this approach during all stages which included the research, design and implementation. In the initial stages of the project I was considering what methodology to use but ultimately decided on using agile as it is what is used by a lot of companies at the moment and I was interested in trying to replicate their approach to developing software.

There are different styles of agile development such as KanBan or extreme programming, but the one I chose for my approach to the project was scrum. Scrum is an agile methodology in which a cycle is carried out in what are known as sprints. 

Throughout the life cycle of the project I held weekly meetings with my project advisor. In these meetings we would discuss what I had done in the previous week and what I was planning to achieve in the following week. Whatever was decided at these meetings, I then created cards on the GitHub project board on the projects page. This helped me focus on the task at hand and I found using this project board to be very useful in the development of this project. Once I started the particular task I would edit the project board and move the card to the in progress column and upon completion of the task move it to the completed section.  

\section{Version Control}
Throughout the life cycle of the project I used GitHub for the version control. GitHub is a hosting service for version control. What I did was created a repository on GitHub and during the development of the project I would commit whatever progress I had made during that session to GitHub to record my progress. 

Tracking the source code I was committing to GitHub was not the only service I used, I also used the task management tool. This allowed me to create tasks that I wanted to complete and assign them to various sections such as to-do or in progress. 

GitHub also provided an opportunity for me to share my work with my project advisor in weeks I could not attend the project meeting. On the repository you can see when new things are committed and what exactly was updated. Another feature I used with GitHub was the ability to rollback the project if needed. Github Tracked the entire progress of the project and at any time I could revert back to any of the previous commits and this proved very useful.

\section{Testing}
For my application I needed to decide how I was going to test the application as I could not use something such as J-unit. In the end I decided to go with white and black box testing. We had looked at this method of testing in a previous module and I thought it was fit the purposes  of testing that I needed.

\subsection{White Box Testing}
In white box testing the tester can see in internal workings of the software. White box testers have a full and comprehensive knowledge of the internal makeup of the software and are usually software developers themselves. This is a role I thought I suited and so I carried out the software tests. What I did was I wrote out some features I would like to test and the expected outcome. For example if I was testing the log-in system, the expected outcome would be that the user is logged in and is presented their user page. I would then test this and follow the path the program follows as it moves down through the code.

\subsection{Black Box Testing}
Black box testing is a way of testing a piece of software without knowing the internal functionality of the software being tested. A black box tester has no knowledge of the internal design, structure or implementation of the software and are often not software developers. For this I asked some of my friends to test out certain features of the software. The web application was hosted on the Google cloud platform and so I was able to send them a link and ask them to try and perform certain functions. Again I would have made a list of expected outcomes and asked them to recommend any changes if they found any problems themselves. I found this very useful as I often skipped over something small from being so used to looking at it the whole time.

\section{Sprints}
In this section I will be talking about sprints. Each of the sprints I tried to complete in a timely manner and as effectively as possible. Each sprint represents a vital part of the application.

\subsection{Sprint 1 Research}
This was my first sprint and I used it to research existing exchanges. I focused on understanding what was user-friendly and how I could potentially improve on existing solutions.

\subsection{Sprint 2 Design}
During this sprint I focused on the design of the application. Not the aesthetic design but rather the database structure and web application structure.

\subsection{Sprint 3 Database Connection}
Once the design was complete it was time to start the first of the implementation sprints. For this sprint the objective was to set up a basic three-tier solution. For this I stored some data in a database and using php I presented it on a web page. 

\subsection{Sprint 4 Log-in and Registration}
Once the basic pipeline was implemented I then decided to work on a log-in system. This log-in system would be pretty comprehensive in regards to the validation checks, cleaning the data before storage in the database and creating a unique session for each user. 

\subsection{Sprint 5 Website Design}
Once the log in system and registration system was implemented I then decided to work on the general website design. I created a home page where a user could log in from and a user page available only to the logged in user and also general website navigation.

\subsection{Sprint 6 User Validation}
This was the sprint where I finished off the user validation I added a system to email the user an activation code when registering and also an activation code when trying to reset passwords. All of these help with the security of the site.

\subsection{Sprint 7 Sessions}
During this Sprint another feature I added in this, was the remember me function. This used session data to remember the user from a particular session and not require the user to log in providing the session has not ended.
 
\subsection{Sprint 8 Top up account}
During this sprint I wanted to add functionality to the user to be able to top up their account have have credit to use on betting/laying events.

\subsection{Sprint 9 Display Data from Database}
This was a very important sprint as it was the one where I pulled the data on the events and presented them to the user so that they could then place a bet or lay on the event.

\subsection{Sprint 10 Place a Bet/Lay}
Now that the data was being displayed, during this sprint I allowed the user to place bets or lays on their selected event. This bet/lay would then be assigned to their account so that the user could see what bets/lays they have placed.

\subsection{Sprint 11 Python Scraper}
The final sprint was to implement a python scraper that would be used to populate the database for the events. This script would be running on the web server and continually updating the data that was being presented to the user.

\chapter{Technology Review}

\section{Data Tier}
\subsection{MySQL}

\section{Logic Tier}
\subsection{Php}
\subsection{Python}


\section{Presentation Tier}
\subsection{HTML}
\subsection{Ajax}

\section{Additional Languages}
\subsection{CSS}
\subsection{Bootstrap}
\subsection{JavaScript}
\subsection{LaTeX}

\section{Technologies Used}
\subsection{Sublime Text}
\subsection{GitHub}
\subsection{Google Cloud Services}
\subsection{WAMP}
\subsection{Overleaf}
\subsection{Cmder}

\chapter{System Design}

\section{Data Tier}
\subsection{MySQL}

\section{Logic Tier}
\subsection{Connection to Data Tiers}
\subsection{Connecting to Presentation Tier}
\subsection{Routing and Accessing data}


\section{Presentation Tier}
\subsection{Website Structure}
\subsection{Pages}
\subsection{Deployment and Hosting}

\chapter{System Evaluation}
\section{Robustness}

\section{Testing}
\subsection{Device Testing}

\section{Limitations}
\subsection{Service Limitations}

\section{Results}

\chapter{Conclusion}
\section{Future Development}
\subsection{App Development}

